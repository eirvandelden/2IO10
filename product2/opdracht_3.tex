%% Heroefening ``Wat een werk'', Opdracht 3

\item Hoelang duurde de langste klus? En welke klus duurde zolang? \\

%\emph{Interpretatie:}  \\
%\emph{Aanname:}  \\

\emph{Antwoord:} De langst klus is $[k \in \mathcal{K}, l \in
\mathcal{L} : ( \uparrow e(l) - \downarrow b(l) ) \wedge k = n(l) :
k ]$ en die klus duurde $[k \in \mathcal{K}, l \in \mathcal{L} : k =
n(l) : ( \uparrow e(l) - \downarrow b(l) ) ] $. \\
\emph{Assessment:} Fouten in syntax.  \\

%\emph{Verbeterd antwoord:} De langste klus duurde $ ( \uparrow l \in
%\mathcal{L} : : ( e(l) - b(l) ) ) $ en dit was de klus $ ( l \in
%\mathcal{L} : ( e(l) - b(l) )  = ( \uparrow l' \in \mathcal{L} : : ( e(l') - b(l') ) %) : n(l) ) $ \\
%\emph{Assessment:} Fouten in syntax.  \\

\emph{Verbeterd antwoord:} De langste klus duurde $ ( \uparrow l \in
\mathcal{L} : : ( e(l) - b(l) ) ) $ en dit was een klus in $ \{ l
\in
\mathcal{L} : ( e(l) - b(l) )  = ( \uparrow l' \in \mathcal{L} : : ( e(l') - b(l') ) ) : n(l) \} $ \\
