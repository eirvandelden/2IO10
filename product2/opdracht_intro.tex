%% De opdracht

Een werkeenheid (denk aan een fabriek of een computerpark) verricht werkzaamheden en houdt daarvan een historie $\mathcal{L}$ (van logfile) bij. Eenvoudige eisen zijn:

\begin{enumerate}
    \item[\textbf{a}] De werkeenheid kan hoogstens A klussen tegelijk aan.
    \item[\textbf{b}] Dezelfde klus kan niet meermalen tegelijk worden uitgevoerd.
\end{enumerate}

De logfile is een verzameling van zogenaamde klusuitvoeringen die de informatie over die
uitvoeringen bevat, zoals de naam van de klus en de begin- en eindtijd van haar uitvoering. Die
informatie kan verkregen worden door geschikte functies los te laten op de klusuitvoeringen:
\begin{center}
    $n : \mathcal{L} \rightarrow \mathcal{K}$ en $b,e : \mathcal{L} \rightarrow \mathbb{T} $
\end{center}
Waarbij het type $\mathcal{K}$ de collectie klussen vertegenwoordigt en $\mathbb{T}$ de tijd.
Aanroepen van die functies zien er dus bijvoorbeeld uit als
\begin{center}
    $n(l) =$ plak, $b(l) =$ 481, $e(l) = 154 * \pi$
\end{center}
De klus plak is dus betrokken bij de uitvoering l en wel gedurende de tijd die gegeven wordt door
het halfopen tijdsinterval $[481, 154 * \pi)$. Je zou kunnen besluiten te denken dat een
klusuitvoering er dus uitziet als een drietal, maar dat is iets te eenvoudig. Zo'n klusuitvoering
ziet er natuurlijk uit als een regel op een kleitablet met allemaal merkwaardige tekens, eenheden
en vreselijk veel andere informatie er doorheen. De functies zijn informatie-extraherende
machientjes, maar dat hoeven we allemaal niet te weten als we niet willen nadenken over de
kleitabletimplementatie.
\newline
\newline
\textbf{Opgave:}
