\documentclass[11pt]{article}
\usepackage{geometry}
\geometry{a4paper}
\usepackage[dutch]{babel}
\usepackage{a4wide}

\title{Waarom willen wij Teerheer als specificatieopdracht?}
\author{OGO 1.1 Groep 1a, \\
R. A. J. Berkeveld}
\date{20 November 2006}

\begin{document}

\maketitle

\section{Achtergrond}
De opdracht Teerheer is een specificatieopdracht voor het OGO 1.1 project. Dit houd in een variant
van Pacman te specificeren. De opdracht is ons bij voorbaat aangeleverd met de gedachte dat wij
hier mee verder zouden gaan in het vervolg van het vak.

Nu heeft het echter zo geschied dat ons eerste productdocument niet door de keuring kwam en wij dus
moesten afzien van deze opdracht. Dit werd zelfs vervangen door een (nog onbekend) alternatief. Dit
was geheel tegen onze verwachtingen in. Dit document dient ter overtuiging van het ons Teerheer als
nog te laten specificeren, en wel grotendeels in de collegevrije week van dinsdag 2 t/m vrijdag 5
Januari.

\section{Praktische vordering}
Op het moment hebben wij al een redelijk idee tot aanpak, de speelomgeving is al logisch geschapen
maar dient nog daadwerkelijk gespecificeerd te worden. We hebben al idee\"{e}n tot het realiseren
van animerende objecten, beweging, collisions, AI, IO en dergelijke. Dit zijn dan wel niet meer dan
brainstorms, maar toch zeker wel implementeerbare gedachten.

Ook zijn wij in beschikking van een (naar ons gelang) zeer creatieve naam voor de resulterende
kloon: Taaru no Kami (Japans voor God van het Teer, ofwel Teerheer).

Dit alles had geschied voordat wij te horen kregen dat we deze opdracht waarschijnlijk niet meer
zouden doen. Vervolgens is dit uit vrees van tijdsverspilling ook niet meer in overweging genomen.
Aldus tot nu, want nu is Productdocument 2 afgerond en zijn wij dus toe aan desbetreffende
opdracht.

\section{Andere redenen}
\begin{itemize}
\item Het uitvoeren van deze opdracht in bovengenoemde week zorgt dat wij dit niet in de
zomervakantie hoeven in te halen.
\item Bovendien zorgt deze oplossing ervoor dat we direct aan de slag kunnen, zonder eerst te hoeven
wachten op een bekendmaking van de nieuwe opdracht. Dit zelfs al v\'{o}\'{o}r betreffende
projecttijd in gaat. Dit bespaart ons tijd die wij in die enkele week tekort kunnen komen.
\item Het bespaart de opdrachtgever natuurlijk ook het werk een nieuwe opdracht te schrijven.
\end{itemize}

\end{document}
