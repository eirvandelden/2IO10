Een gebeurtenis kan men opvatten als een globaal evenement wat gevolgen heeft voor objecten anders dan het eventuele object wat deze gebeurtenis instantieert.
Een gebeurtenis kan ge\"instantieert worden wanneer voldaan wordt aan de eventueel gegeven condities, of een object kan een gebeurtenis aanroepen.

\subsection{Explosie}
Een \textbf{explosie} heeft toepassing op alle bewegende objecten aanwezig op de positie behorend bij die explosie.

\subsection{Aardbeving}
Een \textbf{aardbeving} heeft toepassing op alle samurai.

\subsection{Botsen}
Een \textbf{botsing} vind plaats wanneer 2 objecten op dezelfde positie bevinden, en dan wel op die 2 objecten.

\subsection{Spelstap}
Een \textbf{spelstap} heeft toepassing op alle objecten en vind voor al die objecten gelijktijdig plaats.

\subsection{Initialisatie}
De initialisatie treed op aan het begin van het spel.

\subsection{Winnen}
Winnen heeft toepassing op het spel, dit treedt op wanneer er geen sushi's meer aanwezig zijn op het speelveld.

\subsection{Verliezen}
Verliezen heeft toepassing op het spel, dit treedt op wanneer het aantal levens van Taaru no Kizoku op 0 wordt gebracht.
