% Productdocument van OGO 1.1 goep 1a. Jaargang '06-'07, trimester 1.
% Extra opgaven, getiteld: ``Taaru no kizoku'' (貴族タアルノキゾク)

\documentclass[11pt]{report}

%% Gebruikte pakketten
\usepackage[dutch]{babel}
\usepackage{geometry}
\usepackage{graphicx}
\usepackage{amssymb}
\usepackage{amsmath}
\usepackage{latexsym}

\usepackage{epstopdf}
\usepackage{textcomp}
\usepackage[]{hyperref}
\usepackage{fancyhdr}

%% Documentinstellingen
\geometry{letterpaper} \pagestyle{fancy}
% with this we ensure that the chapter and section
% headings are in lowercase.
\renewcommand{\chaptermark}[1]{\markboth{#1}{}}
\renewcommand{\sectionmark}[1]{\markright{\thesection\ #1}}
\fancyhf{} % delete current setting for header and footer
\fancyhead[LE,RO]{\bfseries\thepage} \fancyhead[LO]{\bfseries\rightmark} \fancyhead[RE]{\bfseries\leftmark}
\renewcommand{\headrulewidth}{0.5pt}
\renewcommand{\footrulewidth}{0pt}
\addtolength{\headheight}{0.5pt} % make space for the rule
\fancypagestyle{plain} {
    \fancyhead{} % get rid of headers on plain pages
    \renewcommand{\headrulewidth}{0pt} } % and the line
\DeclareGraphicsRule{.tif}{png}{.png}{`convert #1 `dirname #1`/`basename #1 .tif`.png}

\topmargin 0.0cm  \textwidth 6.0in \oddsidemargin 0.25in \evensidemargin 0.25in \textheight 8.2in \headheight 0.0cm \headsep 1.0cm \topskip 0.0cm

%% Commando's (overgenomen uit Macrogo.tex)
\newcommand{\cat}{+\!\!\!\!+\:}
\newcommand{\inr}{< \!\!\!\!\!\! - \:}
\newcommand{\oo}{{\scriptstyle{\:\circ}\:}}
\newtheorem{thm}{Theorem}[section]
\newtheorem{opg}[thm]{Opgave}
\newcommand{\catt}{+\!\!\!+\:}

%% Het Document
\begin{document}

%%%%%%%%%%%%%%%%%%%%%%%%%%%%%%%%%%%%%%%%%%%%%%%%%%%%%%%%%%%%%%%%%
% Contents: The title page
% $Id: title.tex,v 1.2 2003/03/19 20:57:47 oetiker Exp $
%%%%%%%%%%%%%%%%%%%%%%%%%%%%%%%%%%%%%%%%%%%%%%%%%%%%%%%%%%%%%%%%%
\ifx\pdfoutput\undefined % We're not running pdftex
\else
\pdfbookmark{Title Page}{title}
\fi
\newlength{\centeroffset}
\setlength{\centeroffset}{-0.5\oddsidemargin}
\addtolength{\centeroffset}{0.5\evensidemargin}
%\addtolength{\textwidth}{-\centeroffset}
\thispagestyle{empty}
\vspace*{\stretch{1}}
\noindent\hspace*{\centeroffset}\makebox[0pt][l]{\begin{minipage}{\textwidth}
\flushright
{\Huge\bfseries Een sprong in\\
Mathematische Hulpmiddelen\\

}
\noindent\rule[-1ex]{\textwidth}{5pt}\\[2.5ex]
\hfill\emph{\Large OGO 1.1: Groep 1a}
\end{minipage}}

\vspace{\stretch{1}}
\noindent\hspace*{\centeroffset}\makebox[0pt][l]{\begin{minipage}{\textwidth}
\flushright
{\bfseries
door
     Leroy Bakker 0617167 \\
     Roy Berkeveld 0608170\\
     Giso Dal 0615787\\
     Etienne van Delden 0618959\\
     Edin Dudojevi\"c 0608206 \\
     Nick van der Veeken 0587266\\ [3ex]}
4 Oktober, 2006
\end{minipage}}

%\addtolength{\textwidth}{\centeroffset}
\vspace{\stretch{2}}



\endinput
 %% Veranderen naar PDF voor japanse tekens

\tableofcontents


\chapter{Samenvatting}
%% Samenvatting aan het begin van het document

Dit document bevat, naast de standaardonderdelen van het verslag, de
antwoorden op de opgaven uit de handout ``Heroefeningswerk''. Het
maken van de opgaven heeft inzicht verschaft in de leer van het
formeel specificeren en het interpreteren van formele specificaties.
Ook heeft het ons in aanraking gebracht met SVN, wat onze werkwijze
een stuk efficienter heeft gemaakt.\\
In dit document zul je opgaven vinden die door de hele groep is gemaakt en eventueel ook verbeterd. Daarnaast hebben we ook
elkaar globaal beoordeeld over het project.


\chapter{Inleiding}
%% Inleiding tot het document

Tijdens dit OGO project is er weer gewerkt aan het specificeren van formules en tekstuele beschrijvingen ervan. De kennis die bij
de regulieren vakken is vergaard, kan in dit project ook daadwerkelijk worden toegepast. Daarnaast is er geleerd om beter in
groepsverband te werken en samen tot een oplossing te komen. Hier gaat meer tijd in zitten dan welk ander onderdeel dan ook. Het
is niet altijd even vlekkenloos verlopen om elkaar te overtuigen van een mening, maar
het team verband is als gevolg van deze opdracht enorm versterkt.\\

Onze technische vaardigheden zijn ook zeker op de proef gesteld als
het gaat om het werken met LaTex. Hoewel Latex niet zo nieuw meer is
voor de groep, is SVN dat wel. Dit document is volledig in elkaar
gezet met behulp van SVN en daarom bijna ondenkbaar, aangezien het
project veel efficienter is verlopen dan voorheen. \\

Vergaderen, notuleren en plannen waren ook zeker belangrijke vaardigheiden die nodig waren om het project succesvol te laten
eindigen. Dit blijkt ook wel uit het verslag \ldots


\chapter{User Requirements}

``Taaru no Kizoku'' is een geschikt spel voor jong en oud! In deze,
japans geori\"enteerde, single player pacmanvariant zul je de
avonturen beleven van sumo worstelaar ``Taaru no Kizoku''. Je zult
jezelf moeten manoeuvreren door de levels, door alle ``sushi's'' te
eten die zich in dat level bevinden. Zorg dat je hierbij niet gepakt
wordt door de gevaarlijke samurai, want die zullen ervoor zorgen dat
een van je levens verliest ,die je in het begin van het spel krijgt.
Dan begint het level dus gewoon opnieuw. De enige restrictie in het
spel is, dat je niet door de muren heen kan gaan.

Er zijn verschillende manieren om het een stukje gemakkelijker
maken. Eet ``Wasabi'' en je zult tijdelijk de kracht weten te vinden
om de samurai te verslaan \ldots Je zult namelijk in brand staan en
bij aanraking met een samurai zul je ze verslaan. Als je genoeg
sushi eet zul je beschikken over verschillende ``power up's''. Hoe
meer je eet, hoe beter, aangezien je ``powers'' steeds krachtiger
worden. Het verzamelen van je energie kun je volgen in de vorm van
potions. Is het eerste flesje gevuld, dan kun je kiezen om hem te
gebruiken of om door te sparen om een betere power op een later
moment te kunnen gebruiken. De eerste en tweede power hebben met
elkaar te maken. Het zijn namelijk bommen die je kunt neerleggen,
met het subtiele verschil dat de tweede bom krachtiger is en dus een
groter bereik heeft. De bom explodeert als er een timer is afgelopen
of als iemand er overheen loopt. Dit wil dus zeggen dat de speler
ook door zijn eigen bom kan worden verslagen. Met de derde power kun
je het speelveld doen laten schokken door een sprong te maken. Het
gevolg hiervan is dat de samurai tijdelijk gedesori\"enteerd zijn en
stil blijven staan. Maar pas op! Als je een power gebruikt, maak je
een hoeveelheid potions op, afhankelijk van welke power je gebruikt.
Je krijgt meer punten voor het verslaan van de samurai ( de
hoeveelheid samurai is afhankelijk van de grootte van het speelveld
) dan door het eten van de sushi, maak hier dus dankbaar gebruik
van. Eet ze!

%Als je genoeg mana potions hebt verzameld ( het maximum ),
%krijg je een tijdelijk extra leven dat alleen geldig is zolang je
%beschikt over dat aantal potions. Deze power wordt automatisch
%gebruikt als je in contact komt met een samurai, waardoor je niet
%opnieuw hoeft te beginnen maar juist door kan spelen.

%Je wordt opnieuw gespawd in het speelveld in met de resterende sushi
%nog aanwezig.



\newpage

\chapter{Inventarisatie}
Voor onze inventarisatie onderscheiden we de volgende 3 elementen: Het spel, het speelveld en de objecten. Deze worden hieronder nader verklaard.

\section{Spel}
Het \textbf{spel} is het hoofdelement van het spel, de rest is hieraan gekoppeld.

Eigenschappen:
\begin{enumerate}
  \item een rij levels, die wij \textbf{speelveld}en noemen
  \item een \textbf{taaru}
  \item een \textbf{index} voor het huidige \textbf{speelveld}

\end{enumerate}
Acties:
\begin{enumerate}
  \item kan \textbf{index verhogen}
  \item kan het spel \textbf{starten}
  \item kan het spel \textbf{stoppen}
\end{enumerate}


\section{Speelveld}
%Het speelveld bestaat uit een rechthoek onderverdeeld in vierkante hokjes van gelijke grootte die het totale oppervlak helemaal opvullen. De rechthoek heeft
%een x-aantal eenheden breedte en een y-aantal eenheden lengte, waarbij ieder hokje uit 1 breedte-eenheid en 1 lengte-eenheid bestaat. Het speelveld is opgevuld
%met allerei objecten.

% in mijn mening is onderstaand duidelijker (roy)
Het speelveld bestaat uit een rechthoekig begrensd gelijkmatig verdeeld raster. Ieder vlak uit dat raster geeft een x- en y-co\"ordinaat aan. Objecten zijn door
middel van deze x- en y-co\"ordinaten gepositioneerd op dit raster.

Eigenschappen:
\begin{enumerate}
  \item \textbf{hoogte} en \textbf{breedte} van het veld (zijn naar onder begrensd, eindig en zijn altijd oneven)
  \item heeft speelveld\textbf{co\"ordinaten}  met een onderlimiet linksonder, oplopend tot (breedte, hoogte) rechtsboven
  \item bevat \textbf{muren}
  \item bevat een \textbf{looppad}
  \item bevat een \textbf{startpositie}


\end{enumerate}





\section{Objecten}

%\includegraphics[width=20cm,angle=90]{overerving_objecten}

\subsection{Object}
Een \textbf{object} heeft bepaalde eigenschappen.

Eigenschappen:
\begin{enumerate}
\item Een \textbf{object} heeft een \textbf{positie} op het looppad.
\end{enumerate}


\subsection{Bewegend object}
Een \textbf{bewegend object} erft alle eigenschappen van een \textbf{object}. \\
Extra eigenschappen:
\begin{enumerate}
  \item heeft een \textbf{snelheid}
  \item heeft een \textbf{richting}
\end{enumerate}
Extra acties:
\begin{enumerate}
  \item kan van \textbf{positie} veranderen
  \item kan van \textbf{snelheid} veranderen
  \item kan van \textbf{richting} veranderen
\end{enumerate}

\subsubsection{Taaru}
Het object \textbf{taaru} erft alle eigenschappen en acties van een \textbf{bewegend object}. \\
Eigenschappen:
\begin{enumerate}
  \item heeft een toestand: \textbf{flaming}
  \item heeft een \textbf{flamingduratie}
  \item heeft een \textbf{score}
  \item heeft een teller voor het aantal gegeten \textbf{sushi}'s: \textbf{sushi\_teller}
  \item een teller voor het aantal \textbf{potions}: \textbf{potion\_teller}
  \item heeft een teller voor het aantal \textbf{levens}: \textbf{levens\_teller}

\end{enumerate}
Acties:
\begin{enumerate}
%\textbf{eten}
  \item kan \textbf{sushi\_teller} veranderen
  \item kan \textbf{potion\_teller} veranderen
  \item kan \textbf{score} veranderen
  \item kan \textbf{flaming} veranderen
  \item kan \textbf{flaming\_duratie} veranderen
  \item kan \textbf{levens\_teller} veranderen

\end{enumerate}

\subsubsection{Samurai}
Samurai erven alle eigenschappen en acties van een \textbf{bewegend object}. \\
Eigenschappen:
\begin{enumerate}
  \item heeft een teller voor het stilstaan: \textbf{stilstaan\_duur}
\end{enumerate}
Acties:
\begin{enumerate}
  \item kan \textbf{verdwijnen} van het spel
  \item kan \textbf{stilstaan\_duur} veranderen
\end{enumerate}

\subsection{Eten}
Eten erft alle eigenschappen van een \textbf{object}. \\
Eigenschappen:
\begin{enumerate}
 \item is een sushi of een wasabi
\end{enumerate}
Acties:
\begin{enumerate}
  \item kan \textbf{verdwijnen}
\end{enumerate}

\subsection{Bommen}
\textbf{Bommen} erven alle eigenschappen een \textbf{object}. \\
Eigenschappen:
\begin{enumerate}
  \item heeft een teller voor het afgaan: \textbf{delay\_teller}
  \item is een kleine\_bom of een grote\_bom
\end{enumerate}

Acties:
\begin{enumerate}
  \item kan \textbf{delay\_teller} veranderen
  \item kan \textbf{verschijnen}
  \item kan \textbf{verdwijnen}
\end{enumerate}

\subsection{Explosie}
\textbf{Explosie} erft alle eigenschappen van \textbf{object}.
Eigenschappen:
\begin{enumerate}
  \item heeft een explosie duratie: \textbf{explosie\_duur}
\end{enumerate}
Acties:
\begin{enumerate}
  \item kan \textbf{verschijnen}
  \item kan \textbf{verdwijnen}
  \item kan \textbf{explosie\_duur} veranderen
\end{enumerate}

\subsection{Barak}
De \textbf{barak} erft alle eigenschappen van een \textbf{object}. \\
Eigenschappen:
\begin{enumerate}
  \item heeft een \textbf{teller}
\end{enumerate}
Acties:
\begin{enumerate}
  \item kan \textbf{teller} veranderen
\end{enumerate}


%\subsection{Eigenschappen}
%\input{Eigenschappen}

\section{Flow van het spel}
\subsection{Initialisatie}
Voor het spel kan beginnen moet er een speelveld worden gemaakt. De
Initialistie wordt gestart en genereerd een speelveld met daarop een
looppad. Alles wat geen looppad is, is een muur. Het looppad is
hierbij dus nooit doodlopend en er kunnen geen ``eilanden''
ontstaan. Dit wil zeggen dat delen van het speelveld niet afgesloten
kunnen zijn voor ``Taaru no Kizoku''. Een barak wordt geplaatst in
het midden van het veld. Alle overige co\"ordinaten van het looppad
worden nu gevuld met sushi of wasabi.

\subsection{Spel}
Het speelveld is nu gemaakt en het spel kan gaan beginnen. ``Taaru
no Kizoku'' krijgt een startpositie en er worden samurai gecre\"eerd
vanuit hun startposite, de barak. De hoeveelheid samurai is
afhankelijk van de grootte van het speelveld. Taaru no Kizoku krijgt
een bepaald aantal levens die globaal over het spel gelden. Deze
levens kunnen verloren gaan op verschillende manieren. Als Taaru
dood gaat, zal het aantal levens afnemen tot een ondergrens is
bereikt, met als gevolg dat het spel eindigt en er een score zal worden weergegeven.\\

Door middel van de gebruiker kan Taaru no Kizoku zich verplaatsen
over alle co\"ordinaten van het looppad. Taaru kan zich alleen
verplaatsen naar een buur-co\"ordinaat op het looppad vanaf zijn
huidige positie. Door sushi of wasabi te eten ( Taaru bevindt zich
op dezelfde positie als sushi of wasabi ), wordt de sushiteller
verhoogd. De waarde van de sushiteller komt overeen met een aantal
potions. Deze potions kunnen worden gebruikt ten
koste van het uiteindelijk te halen punten. \\

Op ieder gegeven moment kunnen de verkregen potions worden gebruikt
in ruil voor Taaru no Kizoku's powers. De powers zullen een
verschillend aantal potions kosten. Een mogelijkheid is om een bom
neer te leggen, waarvan ook een grotere variant beschikbaar is. Bij
het neerleggen van de bom zal er op hetzelfde moment een aflopende
timer gaan lopen. Op het moment dat de timer eindigt, zal deze
verdwijnen en er zullen er verschillende explosies plaatsvinden
waarvan het aantal en de posities afhankelijk zijn van de grootte
van de neergelegde bom. Ieder bewegend object dat zich in het bereik
van de ontploffing bevindt, zal dood gaan. De bom komt ook tot
ontploffing wanneer er een bewegend object zich op dezelfde plek
bevindt als de neergelegde bom, waarbij de timer automatisch
verdwijnt. Een andere power van Taaru no Kizoku is het laten
ontstaan van een aardbeving. Deze zal direct plaatsvinden bij
activering en zal ervoor zorgen dat de samurai stil zullen staan. Op
dit moment zal er voor alle aanwezige samurai een aflopende timer
starten. Wanneer de ondergrens van de timer is bereikt, gaan de
samurai weer bewegen. Voor samurai gelden dezelfde regels van
verplaatsing als voor Taaru no Kizoku zelf. In het geval dat Taaru
no Kizoku een positie deelt met een samurai, zal Taaru no Kizoku
dood gaan, waarna het level volledig opnieuw wordt gestart. Wanneer
Taaru no Kizoku een Wasabi eet, zal zijn staat veranderen in
``flaming''. Dit heeft hetzelfde gevolg als de power aardbeving
wordt gebruikt, echter wanneer Taaru no Kizoku nu een samurai
tegenkomt, zal de samurai dood gaan en zal Taaru no Kizoku een
bepaald aantal bonus punten krijgen. Wanneer de stilstaan-timer van
de samurai eindigt, zal Taaru no Kizoku zijn toestand terug
verranderen naar normaal. De samurai verdwijnt van zijn huidige
positie en verschijnt na enige tijd weer op zijn startpositie, de barak.\\

Wanneer alle sushi en wasabi uit het level zijn verdwenen, wordt de
score bijgewerkt op het huidige aantal sushi's en begint een nieuw level door door de index van
speelveld te verhogen. Het aantal levens en de score blijven behouden,
de andere eigenschappen van pacman worden teruggezet naar hun initiele toestand.



\newpage

\chapter{Specificatie}

%\section{Verzamelingen}\label{sec:verzamelingen} % (fold)

Hieronder staat een lijstje van de verzamelingen waar wij gebruik van maken. Later wordt uitgelegd hoe de verzameling is gespecificeerd. \\ \ \\
$taaru$ \\
$samurais$ \\
$sushi$ \\
$wasabi$ \\
$kleine\_bommen$ \\
$grote\_bommen$ \\
$speelveldco$\"o$rdinaten $ \\
$muren$ \\
$looppad$ \\
$barak$\\
$bewegend$\\
$eten$ \\
$bommen$ \\
$spel$ \\
$objecten$ \\
$explosies$ \\
\\
\\
Wij hebben gebruik gemaakt van de operator $\oplus$ om het variabel zijn van een object aan te duiden. Deze defineren wij als volgt: \\
\\
object' = object $\oplus$ (gegeven\_object, nieuwe\_waarde) \\
waarbij object' op een gekozen moment kan veranderen naar nieuwe\_waarde \\
of, bij een verzameling: \\
$objectverzameling'$ = $objectverzameling$ $\oplus$ (gegeven\_object, nieuwe\_waarde) \\
waarbij het object aangegeven door gegeven\_object uit de verzameling $objectverzameling$ nieuwe\_waarde krijgt, dit maakt dus de $objectverzameling$ dynamisch \\
\\
In principe is dit een flexibele assignment, met het subtiele verschil dat de verandering plaats k\'an vinden, en dit niet onmiddelijk doet.


% section verzamelingen (end)

\section{Spel}\label{sec:spel} % (fold)

$spel = \{speelveld, taaru, index\}$ \\
\begin{enumerate}
\item $index \in \mathbb{N}$
\item $index\_verhogen: \mathbb{N} \rightarrow \mathbb{N}$ \\
      $index\_verhogen(index) = (index')$ \\
      $index' = index \oplus (index, \emph{i})$, met $\emph{i} \in \mathbb{N}$
\item $starten: begin \rightarrow spel$
\item $stoppen: spel \rightarrow score\_teller$
\end{enumerate}

% section spel (end)

\section{Speelveld}\label{sec:speelveld} % (fold)


\begin{enumerate}

 \item hoogte, breedte $ ~ \in \{ x \in \mathbb{N} : x \geq 3 \wedge x~
       \textbf{mod} ~2 = 1: x \}$
 \item $speelveldco$\"o$rdinaten = \{ (x,y) \in \mathbb{N}^2 : x
\leq$ breedte $ ~ \wedge ~ y \leq $ hoogte $ : (x,y) \} $
 \item $buur: \mathbb{N} \times \mathbb{N} \sim \mathbb{N} \times \mathbb{N} $ \\
       $(x_{1},y_{1})buur(x_{2},y_{2}) \Leftrightarrow (x_{1} - x_{2})^2 + (y_{1} - y_{2})^2 = 1$
\item $looppad =  \{ (x_{1}, y_{1}) | \exists _{x_{2}, y_{2}} [ (x_{2}, y_{2}) \in looppad | \exists _{n} [ n : n \in \mathbb{N}^+ : (x_{1},y_{1})buur^n(x_{2},y_{2}) ] ] \} $
 \item \emph{muren} = \emph{speelveldco\"ordinaten} $\setminus \emph{looppad}$
 \item startpositie $\in$ \emph{looppad}
\end{enumerate}
% section speelveld (end)

\section{Objecten}\label{sub:looppad} % (fold)
\begin{enumerate}
 \item $objecten = bewegend \cup eten \cup bommen \cup explosies \cup barak$ \\
\end{enumerate}

In de volgende specificaties wordt bij bepaalde functies gebruik gemaakt van <object>.<eigenschap>, waarbij <object> is meegegeven bij de functie en <eigenschap> deel uitmaakt van de gespecificeerde tupels van het <object>. Bijvoorbeeld positie.x in de eerstvolgende functie. Positie is meegegeven en bestaat uit een x- en y-co\"ordinaat die door positie.x en positie.y kunnen worden gebruikt.

\subsection{Bewegend object}\label{sec:beweging} % (fold)


\emph{bewegend} = \{ (x, y) $|$ x = ( bewegend $\lor$ stilstaand ) $\land$ \\  \indent \indent \indent \indent y = ( noord $\lor$ oost $\lor$ zuid $\lor$ west )\} \\ \\
loop\_naar\_noord: $looppad \rightarrow looppad$ \\
loop\_naar\_noord(positie) = (positie') \\
positie' = positie $\oplus$ (positie, (x,y) ), met: \\
(x,y) = (positie.x, (positie.y)+1) als (positie.x, (positie.y)+1) $\in$ \emph{looppad} en anders (x, y) = (positie.x, positie.y) \\ \\
loop\_naar\_oost: $looppad \rightarrow looppad$ \\
loop\_naar\_oost(positie) = (positie') \\
positie' = positie $\oplus$ (positie, (x,y) ), met: \\
(x,y) = ((positie.x)+1, positie.y) als ((positie.x)+1, positie.y) $\in$ \emph{looppad} en anders (x, y) = (positie.x, positie.y) \\ \\
loop\_naar\_zuid: $looppad \rightarrow looppad$ \\
loop\_naar\_zuid(positie) = (positie') \\
positie' = positie $\oplus$ (positie, (x,y) ), met: \\
(x,y) = (positie.x, (positie.y)-1) als (positie.x, (positie.y)-1) $\in$ \emph{looppad} en anders (x, y) = (positie.x, positie.y) \\ \\
loop\_naar\_west: $looppad \rightarrow looppad$ \\
loop\_naar\_west(positie) = (positie') \\
positie' = positie $\oplus$ (positie, (x,y) ), met: \\
(x,y) = ((positie.x)-1, positie.y) als ((positie.x)-1, positie.y) $\in$ \emph{looppad} en anders (x, y) = (positie.x, positie.y)

\subsubsection{Taaru}\label{sub:taaru_no_kizoku} % (fold)

\emph{taaru} = \{ (x, positie, flaming, flaming\_duratie, score\_teller, sushi\_teller, potion\_teller, levens\_teller) \begin{math} | x \in \emph{bewegend} \land  positie \in \emph{looppad} \land \\
  \indent \indent \indent \indent flaming \in \mathbb{B} \land  flaming\_duratie \in \mathbb{N} \land \\  \indent \indent \indent \indent score\_teller \in \mathbb{N} \land sushi\_teller \in \mathbb{N} \land \\  \indent \indent \indent \indent potion\_teller \in \mathbb{N} \land levens\_teller \in \mathbb{N} \}
\end{math} \\

Acties:
\begin{enumerate}
    \item verander flaming: $\mathbb{P}(taaru) \times \mathbb{B} \rightarrow \mathbb{P}(taaru) $ \\
verander flaming($taaru$, bool) = ($taaru$') \\
$taaru$' = $taaru$ $\oplus$ ($taaru$, ( taaru.x, taaru.positie, bool, taaru.flaming\_duratie, \ldots, taaru.levens\_teller)), met $bool \in \mathbb{B}$ \\
    \item verander flaming\_duratie:  $\mathbb{P}(taaru) \times \mathbb{N} \rightarrow \mathbb{P}(taaru) $ \\
    verander flaming\_duratie($taaru$, duratie) = ($taaru$') \\
    $taaru$' = $taaru$ $\oplus$ ($taaru$, ( taaru.x, taaru.positie, taaru,flaming, duratie, taaru.score\_teller, \ldots, taaru.levens\_teller)), met $duratie \in \mathbb{N}$ \\
     \\  \item verander score\_teller: $\mathbb{P}(taaru) \times \mathbb{N} \rightarrow \mathbb{P}(taaru)$ \\
    verander score\_teller($taaru$, score) = ($taaru$') \\
    $taaru$' = $taaru$ $\oplus$ ($taaru$, ( taaru.x, \ldots , taaru.duratie, score, taaru.sushi\_teller \ldots, taaru.levens\_teller)), met $score \in \mathbb{N}$ \\

\item verander sushi\_teller: $\mathbb{P}(taaru) \times \mathbb{N} \rightarrow \mathbb{P}(taaru)$ \\
    verander sushi\_teller($taaru$, sushi) = ($taaru$') \\
    $taaru$' = $taaru$ $\oplus$ ($taaru$, ( taaru.x, \ldots , taaru.score, sushi, taaru.potion\_teller, taaru.levens\_teller)), met $sushi \in \mathbb{N}$ \\
    \item verander potion\_teller: $\mathbb{P}(taaru) \times \mathbb{N} \rightarrow \mathbb{P}(taaru)$ \\
    verander potion\_teller($taaru$, potion) = ($taaru$') \\
    $taaru$' = $taaru$ $\oplus$ ($taaru$, ( taaru.x, \ldots , taaru.sushi\_teller, potion, taaru.levens\_teller)), met $potion \in \mathbb{N}$ \\

    \item verander levens\_teller: $\mathbb{P}(taaru) \times \mathbb{N} \rightarrow \mathbb{P}(taaru)$ \\
    verander levens\_teller($taaru$, levens) = ($taaru$') \\
    $taaru$' = $taaru$ $\oplus$ ($taaru$, ( taaru.x, \ldots , taaru.potion\_teller, levens)), met $levens \in \mathbb{N}$ \\

\end{enumerate}

% subsection taaru_no_kizoku (end)


\subsubsection{Samurais}\label{sec:samurai} % (fold)

\emph{samurais} = \{ (x, positie, stilstaan\_duur) $|$ (x $\in$ \emph{bewegend}) $\land$ (positie $\in$ \emph{looppad}) $\land$ (stilstaan\_duur $\in \mathbb{N})$ \}\\ \\


Acties:
\begin{enumerate}
  \item verdwijn: $\mathbb{P}$(\emph{samurais}) $\times$ \emph{looppad} $\rightarrow$ $\mathbb{P}$(\emph{samurais}') \\
  verdwijn (\emph{samurais}, positie') = (\emph{samurais}') \\
  \emph{samurais}' = \emph{samurais} $\backslash$ \{x, positie, stilstaan\_duur $|$ positie = positie'\}
  \item verander stilstaan\_duur: $\mathbb{P}$(samurais) $\times$ $\mathbb{N}$ $\rightarrow$ $\mathbb{P}$(samurais) \\
  verander stilstaan\_duur (\emph{samurais}, nieuwe\_duur) = (samurais') \\
  \emph{samurais'} = stilstaan\_duur $\oplus$ (samurais, (samurais.x, samurais, positie,  nieuwe\_duur)
\end{enumerate}


% section beweging (end)


\subsection{Eten}\label{sub:eten} % (fold)
\emph{eten} = \{ (soort, positie) $|$ soort = ( sushi $\lor$ wasabi) $\land$ positie $\in$ \emph{looppad} \} \newline

Acties:
\begin{enumerate}
  \item verdwijn: $\mathbb{P}$(eten) $\times$ \emph{looppad} $\rightarrow$ $\mathbb{P}$(eten) \\
  verdwijn (\emph{eten}, eten\_positie) = (\emph{eten}') \\
  \emph{eten}' = \emph{eten} $\oplus$ (eten\_positie, $\emptyset$)

\end{enumerate}
\ \\
\emph{sushi} = \{ (soort, positie) $\in eten |$ (soort = sushi) \} \\
\emph{wasabi} = \{ (soort, positie) $\in eten |$ (soort = wasabi)  \}



% subsection eten (end)



\subsection{Bommen}\label{sec:bom} % (fold)
\emph{bommen} = \{ (soort, delay\_teller, positie) $|$ soort = (klein $\lor$ groot) $\land$ delay\_teller $\in \mathbb{N}$ $\land$ positie $\in looppad$ \} \\
$kleine\_bommen$ = \{ (soort, delay\_teller, positie) $\in bommen |$ (soort = $klein$) \} \\
$grote\_bommen$ = \{ (soort, delay\_teller, positie) $\in bommen |$ (soort = $groot$)  \} \\


Acties:
\begin{enumerate}
 \item verander delay\_teller: $\mathbb{P}(bommen) \times \mathbb{N}  \rightarrow \mathbb{N}$ \\
       verander delay\_teller($bommen$, $i$) = ($bommen$')  \\
       $bommen'$ = $bommen$ $\oplus$ ( $bommen$, ( bommen.soort, $i$, bommen.positie ) ), met $i \in \mathbb{N}$ \\
 \item verschijn: $\mathbb{P}(bommen) \times looppad  \rightarrow \mathbb{P}(bommen)$ \\
       verschijn($bommen$, positie) = ($bommen'$) \\
       $bommen' = bommen \oplus ( bommen, $$(bommen.soort, bommen.delay\_teller, $positie$ )$ \\
 \item verdwijn: $ \mathbb{P}(bommen)\rightarrow \mathbb{P}(bommen)$ \\
       verdwijn( $bommen$ = ($bommen'$) \\
       $bommen' = bommen \oplus ( bommen,$ ( bommen.soort, bommen.delay\_teller, $\emptyset ) )$ \\
\end{enumerate}

% section bom (end)

\subsection{Explosies}\label{sub:explosie} % (fold)

\emph{explosies} = \{ (positie, duur) $|$ (positie $\in looppad) \land$ (duur $\in \mathbb{N}$) \}


Acties:
\begin{enumerate}
 \item verschijn: $looppad \times \mathbb{P}$(explosies) $\rightarrow \mathbb{P}$(explosies) \\
       verschijn(explosie.positie, $explosies$) = ($explosies'$) \\
       $explosies'$ = $explosies \oplus ( explosies, explosies \cup \{ ( $explosie.positie$, $i$ )  \} )$, met $i \in \mathbb{N}$ \\
 \item verdwijn: $looppad\times \mathbb{P}($explosies) $\rightarrow \mathbb{P}($explosies) \\
       verdwijn(explosie.positie, $explosies$) = ($explosies'$) \\
       $explosies'$ = $explosies \oplus ( explosies, explosies \setminus \{  ($explosie\_positie$, $i$) \} )$, met $i \in \mathbb{N}$ \\
 \item verander duur: $\mathbb{P}( explosies ) \rightarrow \mathbb{P}(explosies)$ \\
       verander duur(explosie) = (explosie') \\
       explosie' = explosie $\oplus$ (explosie, ( explosie.positie, $i$) ), met $i \in \mathbb{N}$ \\
\end{enumerate}

\subsection{Barak}\label{sub:barak} % (fold)

\emph{barak} = \{ ( positie, teller ) $|$ positie $\in looppad \land$ teller $\in \mathbb{N}$ \} \\

\begin{enumerate}
    \item verander teller: $barak \rightarrow barak$ \\
 verander teller(barak, $i$) = (barak') \\
 barak' = teller $\oplus$ (barak, (barak.positie, \emph{i})), met  \emph{i} $\in \mathbb{N}$\\
\end{enumerate}


% subsection barak (end)



\chapter{Synthese}
Voor onze opdracht werd ons niet verplicht gesteld om de synthese volledig uit te werken. W\'el werd ons gevraagd om hierover na te denken. \\
\\
Ons lijkt het zeker mogelijk om de hele samenhang van het spel formeel te defineren in een flowchart. Dit werkt goed samen met de acties op objecten zoals wij
die nu hebben geimplementeerd. \\
\\
Een voorbeeld zou een begin en einde bevatten waartussen een initialisatie voor het spel bevind, waarna een loop plaatsvindt waarin een initialisatie van
het speelveld plaatsvindt, samen met het verdere verloop van het spel. Het einde bijvoorbeeld is dan direct gekoppelt aan de situatie waarin het aantal levens 0
is en Taaru een leven verliest.

%voorbeeld?


%\chapter{Conclusie}
%%% Conclusie aan het eind van het verslag

Na het individueel maken van de opdrachten zijn deze door de hele groep gezamelijk gecorrigeerd en
samengevoegd tot de eerste versie van het productverslag. Deze versie is door de tutor van
commentaar voorzien en vervolgens door de groep verbeterd. De meeste fouten zaten initi\"eel in het
gedeelte over relatie-expressies. Na een uitleg door de tutor over dit onderdeel en een frisse,
kritische blik op deze en de andere in eerste instantie nog niet geheel correcte opgaven, zijn de
gecorrigeerde, definitieve antwoorden op papier gezet.



%\chapter{Begrippenlijst}
%%% De begrippenlijst (tekens, verzamelingen, woorden en functies

\section{Tekens}
\begin{description}
    \item[$\in$]{element van}
    \item[$\vee$]{of}
    \item[$\wedge$]{en}
    \item[$\equiv$]{equivalent}
    \item[$\#$ ]{aantal}
    \item[$\exists$]{er is}
    \item[$\forall$]{voor alle}
    \item[$\emptyset$]{lege verzameling}
    \item[$\neg$]{negatie}
    \item[$\uparrow$]{maximum}
    \item[$\downarrow$]{minimum}
    \item[$\subseteq$]{bevat in}
    \item[$\circ$]{compositie}
    \item[$\cup$]{vereniging}
    \item[$\cap$]{doorsnede}
    \item[$\setminus$]{verschil}
    \item[$\times$]{cartesisch product}
    \item[$\inr$]{bewoner van}
    \item[$\catt$]{catenatie}
    \item[$\Delta$]{identiteitsrelatie}
\end{description}


\section{Verzamelingen}
\begin{description}
    \item[$\mathbb{L}$]{lijst} \\
    \item[$\mathbb{P}$]{machtsverzameling} \\
    \item[$\mathbb{F}$]{eindige verzameling} \\
    \item[$\mathbb{R}$]{de standaard verzamelingen ree\"ele getallen} \\
    \item[$\mathbb{Z}$]{de standaard verzamelingen gehele getallen} \\
    \item[$\mathbb{B}$]{de standaard verzamelingen boolse waarden} \\
    \item[$\mathbb{N}$]{de standaard verzamelingen natuurlijke getallen} \\
\end{description}


\section{Woorden}
\begin{description}
    \item[Bulktypen]{Verzamelingen en rijtjes zijn zogenaamde bulktypen, waarin van een gegeven basistype meerdere exemplaren bijeengegaard zijn}
    \item[Verzamelingen]{Verzamelingen zijn bulks met van elk ding ten hoogste 1 exemplaar}
    \item[Rijtjes]{Rijtjes zijn bulks die per ding meerdere exemplaren mogen hebben, waarbij de volgorde belangrijk is}
    \item[Catenatie]{Het aaneenritsen van twee rijtjes}
    \item[Comprehensie]{Selectie en functietoepassing op de bewoners van die rijtjes gebruiken om de nieuwe rij-bewoners te beschrijven}
    \item[Dummies]{Hulpvariabelen}
    \item[Vereniging]{Catenatie van verzamelingen}
    \item[Compositie]{Het toepassen van een bepaalde functie op een argument}
    \item[Quantificatie]{Het beschrijven van een formule door middel van quantoren $\forall$ en $\exists$}
    \item[Quantor]{Een expressie die het domein van een bepaalde term waaraan het vastzit weer geeft}
    \item[Relaties]{Een verzameling van tupels, rijtjes van elementen, ieder uit een van de verzamelingen waarop de relatie gedefinieerd is.}
    \item[Expressies]{Een collectie van symbolen die samenvoegend een quantiteit beschrijven}
    \item[Functies]{Een relatie of expressie dat een of meer variabelen betrekt}
    \item[Predikaten]{$\mathbb{B}$-waardige functies}
    \item[Existenti\"ele quantificatie]{Quantificatie waarbij geldt: ``Er bestaat een x die aan P voldoet zo, dat daarvoor Q waar is.''}
    \item[Universele quantificatie]{Quantificatie waarbij geldt: ``Voor alle x waarvoor P waar is, is Q waar.''}
    \item[Propositie]{Een predikaat zonder variabelen}
    \item[Variabele]{Argumenten van een $\mathbb{B}$-waardige functie}
    \item[Consecutieve rij]{Een aaneengesloten rij}
    \item[Tupel]{groepering van variabelen}
\end{description}


\section{Functies}
\begin{description}
\label{sec:functies}
    \item[$succ$]{geeft de opvolger} \\
    \item[\lbrack$\cdot $ \rbrack]{geeft een rij aan} \\
    \item[\lbrack \rbrack]{een lege rij}
    \item[$\pi_{2}$]{$2^{de}$ in een tupel} \\
    \item[$fib$]{rij van Fibonacci} \\
    \item[$avg$]{gemiddelde van een lijst} \\
    \item[$0^{\bullet}$]{altijd nul} \\
    \item[$dkr$]{dubbel kwadraten rij} \\
    \item[$rev$]{keer een rij om} \\
    \item[$dkv$]{dubbel kwadraten verzameling} \\
    \item[$set$]{de verzameling gemaakt uit een gegeven rij} \\
    % onderstaand is niet meer nodig
    %\item[$seg$]{$\{p, q : 0 \leq p < q < \#W : [i : p \leq i \leq q : W(i)]\} $\\
        %geeft alle consecutieve deelrijen uit de opgegeven rij}\\
    %\item[$palindroom$]{$\forall_{i} [i \in \mathbb{N} : 0  \leq i \leq (\#ar / 2) - 1 \wedge W(i) = W(\# W - i - 1)] ]$ \\
        %boolse functie die controleert op het zijn van een palindroom} \\
\end{description}



% De bibliografie
%%% De bibliografie (verzameling gebruikte werken)

\begin{thebibliography}{99}

\bibitem{HO} Handout Projectwijzer OGO 1.1, 2006-2007
\bibitem{emh} Handout Enige wiskundige hulpmiddelen, 2006-2007
%doen?: \bibitem{} Handout ``Heroefening: Wat een werk''
\bibitem{NSSI2L} Tobias Oetiker, Hubert Partl. Irene Hyna, Elisabeth Schlegl,
    The Not So Short Introduction To Latex, 2006
\bibitem{L&V} Rob Nederpelt, Fairouz Kamareddine, Logical Reasoning: A First
    Course, King's College Publications, London, 2004

\end{thebibliography}



\end{document}
