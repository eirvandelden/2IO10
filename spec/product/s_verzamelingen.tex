%\section{Verzamelingen}\label{sec:verzamelingen} % (fold)

Hieronder staat een lijstje van de verzamelingen waar wij gebruik van maken. Later wordt uitgelegd hoe de verzameling is gespecificeerd. \\ \ \\
$taaru$ \\
$samurais$ \\
$sushi$ \\
$wasabi$ \\
$kleine\_bommen$ \\
$grote\_bommen$ \\
$speelveldco$\"o$rdinaten $ \\
$muren$ \\
$looppad$ \\
$barak$\\
$bewegend$\\
$eten$ \\
$bommen$ \\
$spel$ \\
$objecten$ \\
$explosies$ \\
\\
\\
Wij hebben gebruik gemaakt van de operator $\oplus$ om het variabel zijn van een object aan te duiden. Deze defineren wij als volgt: \\
\\
object' = object $\oplus$ (gegeven\_object, nieuwe\_waarde) \\
waarbij object' op een gekozen moment kan veranderen naar nieuwe\_waarde \\
of, bij een verzameling: \\
$objectverzameling'$ = $objectverzameling$ $\oplus$ (gegeven\_object, nieuwe\_waarde) \\
waarbij het object aangegeven door gegeven\_object uit de verzameling $objectverzameling$ nieuwe\_waarde krijgt, dit maakt dus de $objectverzameling$ dynamisch \\
\\
In principe is dit een flexibele assignment, met het subtiele verschil dat de verandering plaats k\'an vinden, en dit niet onmiddelijk doet.


% section verzamelingen (end)
