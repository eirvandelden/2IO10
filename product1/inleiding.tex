%% Inleiding tot het document

Het doel van OGO is leren samenwerken aan een project dat qua omvang en benodigde tijd veel groter
is dan bijvoorbeeld huiswerk opgaven. De kennis die bij de reguliere vakken is vergaard kan bij OGO
in de ``praktijk'' worden toegepast.

Omdat dit het eerste project is, kan er geen opgedane kennis worden toegepast, dus ligt de nadruk
nu op de training van zogenaamde OGO vaardigheden, zoals vergaderen, notulen maken en plannen.
Daarnaast zullen we leren werken met \LaTeX, om nette, wetenschappelijke documenten te maken.

Het volgende OGO project gaat over specificeren, vandaar dat er nu tijdens de OGO bijeenkomsten
getraind wordt in het opstellen van formele expressies aan de hand van tekstuele beschrijvingen en
het in woorden beschrijven van gegeven formele expressies. Hiervoor hebben wij opgaven gemaakt uit
de het boekje ``Enkele mathematische hulpmiddelen'' \cite{emh}.
