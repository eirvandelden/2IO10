%% productdocument 1, opgave 6 (Predikaten)

\begin{enumerate}

%%%%% Deelopgave 6.1 %%%%%
\item Geef de betekenis van de volgende predikaten weer als een Nederlandse zin \ldots
\begin{enumerate}

    \item $(\forall x,y,z  : xRy\;\land\; xRz : y=z)$ \\
        \emph{Antwoord:} ``Voor alle voorkomens van \emph{y} en \emph{z} die dezelfde relatie met alle voorkomens van \emph{x} handhaven, geldt dat \emph{y} en \emph{z} gelijk zijn aan elkaar.'' \\

    \item $( \exists a,b,c  : a,b,c \in \mathbb{Z} : a^{n} + b^{n} = c^{n} ) \Rightarrow n\leq 2$ \\
        \emph{Antwoord:} ``Er bestaan waarden voor \emph{a}, \emph{b} en \emph{c} (uit $\mathbb{Z}$) waarvoor geldt $a^{n} + b^{n} = c^{n} \Rightarrow n \leqslant 2$.'' \\
        \emph{Verbeterd antwoord:}``Als er waarden zijn voor \emph{a, b} en \emph{c} uit $\mathbb{Z}$ waarbij $a^{n} + b^{n} = c^{n}$, dan is \emph{n} kleiner of gelijk aan 2.'' \\
        \emph{Redenering:} Het origineel had over het hoofd gezien dat de $\exists$ ophield voor de $\Rightarrow$ en had daardoor een verkeerde interpretatie.\\

    \item $(\uparrow m : m \in M : m)~\mbox{bestaat} \equiv (\exists n  : n \in \mathbb{N} : n\!+\!1 = \# M )$ \\
        %orig 1: \emph{Antwoord:} ``De maximale waarde van \emph{m} (uit \emph{M}) is gelijkwaardig met de lengte van de rij \emph{M}, waarvan de lengte wordt gedefinieerd als \emph{n} + 1 en \emph{n} $\epsilon$ $\mathbb{N}$.''\\
        %orig 1: \emph{Verbeterd antwoord:} Wanneer een maximum voor verzameling $M$ bestaat, dan is $M$ geen lege verzameling en vice versa.\\
        %orig 1: \emph{Redenering:} Het ``bestaat'' was gemist en zorgde voor een verkeerde interpretatie. \\
        \emph{Antwoord:} ``Wanneer een maximum voor verzameling $M$ bestaat, dan is $M$ geen lege verzameling en vice versa.'' \\
        \emph{Verbeterd antwoord:} ``Wanneer een maximum voor verzameling $M$ bestaat, dan is $M$ geen lege verzameling en is $M$ eindig door een limiet. Dit geld ook omgekeerd.'' \\
        \emph{Redenering:} Er werd geen rekening gehouden met een mogelijke afwezigheid van een maximum door oneindigheid d.m.v. naderende limieten. \\

\end{enumerate}

%%%%% Deelopgave 6.2 %%%%%
\item Geef predikaten voor \ldots
\begin{enumerate}

    \item een gegeven woord is een palindroom van ten hoogste $13$ letters lang is. \\
        \emph{Antwoord:} $( \forall ~woord : \#woord \geq 13 \wedge woord \in Palindroom : woord ) $ \\
        \emph{Verbeterd antwoord:} $\#woord \leq 13 \wedge (\forall i \in \mathbb{N} : 0 \leq i \leq ((\#woord$ div $2) \wedge woord(i)=woord(\#woord-1-i))$ \\
        \emph{Redenering:} Het begrip Palindroom is nu uitgewerkt, verder is de formule een beetje anders opgezet. \\

    \item als een gegeven woord een palindroom bevat dan heeft dat hoogstens $13$ letters. \footnote{Er zijn meerdere interpretaties mogelijk van deze zin. Wij nemen aan dat gevraagd wordt dat het palindroom hoogstens 13 letters bevat, en niet het woord zelf.} \\
        %orig 2b \emph{Antwoord:} \textbf{if} $woord \in Palindroom \rightarrow \# woord \leq 13 $ \textbf{fi} \\
        %orig 2b \emph{Verbeterd antwoord:} $((\forall i \in \mathbb{N} : 0 \leq i < ((\#woord / 2) - 1)) \wedge woord(i)=woord(\#woord-1-i)) \Rightarrow \# woord \leq 13 $ \\
        %orig 2b \emph{Redenering:} Het begrip ``Palindroom'' is uitgewerkt, verder niets.\\
        \emph{Antwoord:} $((\forall i \in \mathbb{N} : 0 \leq i < ((\#woord / 2) - 1)) \wedge woord(i)=woord(\#woord-1-i)) \Rightarrow \# woord \leq 13 $ \\
        \emph{Verbeterd antwoord:} $ ( deelwoord \in seg(woord) \wedge palindroom( deelwoord) ) \Rightarrow \#deelwoord \leq 13 $ \\
        \emph{Redenering:} Nu wordt gekeken of het woord een palindroom bevat en niet alleen of het een
        palindroom \\
        is. Verder is Palindroom een boolse functie en seg een functie die alle niet lege consecutieve deelrijen van
        een ingegeven rij geeft. \\

    \item van een gegeven woord is geen enkel segment met lengte ten minste $14$ een palindroom. \\
        \emph{Aanname:} \#W $>$ 0 \\
        %orig 2c \emph{Antwoord:} $\exists_{ar} [ar \in W \wedge \#ar \geq 14 : \forall_{ar} [ \neg palindroom(ar)] ]$ \\
        %orig 2c \emph{Verbeterd antwoord:} $\neg (\exists_{ar} [ar \in seg(W) :   \#ar \geq 14 \wedge \forall_{ar} [$palindroom(ar)$]) $ voor de definitie van palindroom en seg, zie sectie \ref{sec:functies} op \pageref{sec:functies}  \\
        %orig 2c \emph{Redenering:} Het begrip ``segment'' en ``palindroom'' zijn hierin geimplementeerd. \\
        \emph{Antwoord:} $\neg (\exists ar [ar \in seg(W) :   \#ar \geq 14 \wedge \forall_{ar} [$palindroom(ar)$]) $ voor de definitie van palindroom en seg, zie sectie \ref{sec:functies} op \pageref{sec:functies}  \\
        \emph{Verbeterd antwoord:} $\forall_{deelwoord} [ deelwoord \in seg(woord) \wedge \#deelwoord > 13 : \neg palindroom(deelwoord) ]$ \\
        \emph{Redenering:}

\end{enumerate}
\end{enumerate}
