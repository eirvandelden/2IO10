% Voortgangstabel productdocument van OGO 1.1 goep 1a. Jaargang '06-'07, trimester 1.

\documentclass[11pt]{article}

\usepackage{geometry}
\geometry{a4paper}
\usepackage[dutch]{babel}
\usepackage{a4wide}
\usepackage{amssymb}
\usepackage{ulem}

\begin{document}

\textbf{Legenda:}\\
T: Toegewezen\\
OK: Gemaakt, klaar voor review\\
Y: Door peer reviewer goedgekeurd\\
N: Door peer reviewer afgekeurd \\

Iedereen vult deze waarden naar eigen voortgang in onderstaande tabel in en commit deze na iedere
wijziging.

Een opdracht is definitief af wanneer de opdracht een OK bevat en \'{a}lle nakijkers een Y
gegeven hebben. Het vakje ``$\square$''  moet dan worden omgezet naar een ``$\boxtimes$''

Een N mag veranderen in een Y na overleg, mocht een N blijven staan dan wordt een \textbf{nieuwe}
regel toegevoegd met de revisie verhoogd. Hier dienen vervolgens opnieuw Y's voor gegeven te
worden.

\hspace{5pt}

\begin{tabular}{l ||  r l | c c c c c}
\textbf{Af}\ & \textbf{Opdr.} & \textbf{Rev.} & \textbf{Leroy} & \textbf{Roy} & \textbf{Giso} & \textbf{Etienne} & \textbf{Edin} \\

\hline

% Om een aangevinkt vakje te maken pas je $\square$ aan naar $\boxtimes$

$\boxtimes$ & 1 & a & OK & Y & Y & Y &   \\ \hline

$\boxtimes$ & 2 & a & Y & OK & Y & Y & Y  \\ \hline

$\boxtimes$ & 3 & a & N & & OK & N & N  \\ \hline

  &  & b & OK & & N & N & N  \\ \hline

  &  & c & Y & & Y & OK & Y  \\ \hline

$\boxtimes$ & 4 & a & N & N & & OK & N  \\ \hline

  &  & b & Y & Y &  & OK & Y  \\ \hline

$\boxtimes$ & 5 & a & N & N & N & & OK  \\ \hline

  &  & b & Y & Y & Y & Y & OK  \\ \hline

$\boxtimes$ & 6 & a & N & N & N & & OK \\ \hline

  &  & b & OK & Y & Y & & OK  \\ \hline

$\boxtimes$ & 7 & a & N & N & & OK & N \\ \hline

  &  & b & Y & Y & & OK & Y  \\ \hline

$\boxtimes$ & 8 & a & N & & OK & N & N  \\ \hline

    &  & b & Y & & OK & Y & Y  \\ \hline

$\boxtimes$ & 9 & a & N & OK & N & N & N  \\ \hline

  &  & b & Y & OK & Y & Y & Y  \\ \hline

$\boxtimes$ & 10 & a & OK & Y & Y & N & Y \\ \hline

$\boxtimes$ & 11 & a & OK & Y & Y & Y &  \\ \hline

$\boxtimes$ & 12 & a & Y & OK &  Y & N & Y  \\ \hline

$\boxtimes$ & 13 & a & Y & & OK & Y & Y  \\ \hline

  &  & b & N & & OK & N & N  \\ \hline

$\boxtimes$ & 14 & a & Y & Y & & OK & Y  \\ \hline

  &  & b & OK & N & & N & N  \\ \hline

$\square$ & 15 & a & N & N & N & & OK \\ \hline

  &  & b & - & - & - & & OK  \\ \hline

$\boxtimes$ & 16 & a & Y & Y & Y & Y & OK \\ \hline


$\boxtimes$ & 17 & a & N & N & & OK & N \\ \hline

  &  & b & Y & OK & & Y & Y \\ \hline

$\boxtimes$ & 18 & a & N & & OK & N & N  \\ \hline

&  & b & Y & Y & OK & Y &   \\ \hline

$\boxtimes$ & 19 & a & N & OK & N & N & N  \\ \hline

  &  & b &  & OK & Y & Y & Y  \\ \hline

$\boxtimes$ & 20 & a & OK & OK & Y & Y &  \\ \hline

$\boxtimes$ & 21 & a & OK & Y & Y & Y &  \\ \hline

$\boxtimes$ & 22 & a & & OK & Y & Y & Y \\ \hline

$\boxtimes$ & 23 & a & Y & & OK & Y & Y  \\ \hline

$\boxtimes$ & 24 & a & N & N & & N OK & N  \\ \hline

  &  & b & Y & Y & Y & OK &   \\ \hline

$\square$ & 25 & a & - & - & - & & OK  \\ \hline

$\boxtimes$ & 26 & a & OK & Y & Y & Y & OK  \\ \hline

$\boxtimes$ & 27 & a & Y & Y & & OK & Y  \\ \hline

$\boxtimes$ & 28 & a & Y & & OK & Y & Y  \\ \hline

$\boxtimes$ & 29 & a & & OK & Y & Y & Y  \\ \hline

$\boxtimes$ & 30 & a & OK & Y & Y & Y & \\ \hline

\end{tabular}
\end{document}
