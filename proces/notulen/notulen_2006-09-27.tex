\documentclass{article}
\usepackage{amssymb}

\title{Notulen vergadering OGO 1.1}
\author{E.I.R. van Delden}

\begin{document}

\maketitle


\section{De vergadering in het algemeen}

Het is woensdag 27 september 2006, 14:35. Dit is de vierde
vergadering van het OGO.

De voorzitter van deze vergadering is Giso Dal, Etienne van Delden is
notulist.

Iedereen van de groep is aanwezig, inclusief onze tutor Martijn.

De vergadering is gesloten op 15:00.


\section{Besproken punten}

\begin{itemize}
\item Over de vorige notulen: Martijn is de tutor, niet de groepsleider. 

\item De peer reviews worden uitgedeeld, deze zullen de volgende vergadering, woensdag 4-10, worden besproken

\item De tweede OGO  opdracht komt ofwel in week 40 of in week 42.

\item De harde deadline voor zowel het product- als het procesdocument is vrijdag 06-10 om 23:59.

\item De volgende vergadering is op woensdag 4-10, om 15:00

\item We worden attent gemaakt op de peer assessments en peer reviews.
Dat wil zeggen het elkaar cijfers geven en het gezamenlijk
controleren van het werk van de groepsleden.
\end{itemize}


\section{Afspraken}

\begin{itemize}
\item Op de planning stond dat maandag, 25-09,  een proef versie van het productdocument afgemaakt zou worden om door Martijn te laten nakijken. Deze deadline is echter niet gehaald en is verlengd naar woensdag 27-09.

\item Er is een extra bijeenkomst van de groep en de tutor, om het ingeleverde productdocument te bespreken. Deze bijeenkomst is gepland op 29-09, om 13:30 in HG8.58.

\item Overige planning word niet gewijzigd.
\end{itemize}

\section{Aflsuiting}

\begin{itemize}
\item Er word afgesloten met een moraal door Leroy Bakker: ``Kort, maar krachtig!''
\end{itemize}

\end{document}
