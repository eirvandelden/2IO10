\documentclass{article}
\usepackage{amssymb}

\title{Notulen vergadering OGO 1.1}
\author{R.A.J. Berkeveld}

\begin{document}

\maketitle


\section{De vergadering in het algemeen}

Het is woensdag 13 september 2006, 14:19. Dit is de tweede
vergadering van het OGO.

De voorzitter van deze vergadering is Leroy Bakker, Roy Berkeveld is
notulist.

Iedereen van de groep is aanwezig, inclusief onze begeleider
Martijn.

Leroy leest de agenda voor (deze was door Nick samengesteld) en
behandeld de punten. Ook de notulen van de vorige vergadering worden
nogmaals doorgenomen, dit op advies van Martijn.

De vergadering sluit op 14:34.


\section{Besproken punten}

\begin{itemize}
\item Bespreking van notulen is nodig volgens Martijn, hier was
iedereen het mee eens.

\item De planning moet duidelijker.

\item Op tijd aanwezig zijn is belangrijk.

\item Er worden logboeken digitaal bijgehouden, en wel op
lms.ogotue.nl.

\item Mogelijkheid tot inleveren van het werk v\'{o}\'{o}r de deadline bij
Martijn. Dan is hier nog mogelijkheid tot controle en verbetering.
Dit geld voor het Productdocument, het procesdocument heeft geen
uitgebreidde inhoudelijke controle nodig.

\item Het productdocument is `voor een bedrijf' en dient volledig
correct en gecontroleerd te zijn. Het procesdocument is voor ons
persoonlijk bedoeld.

\item Stof die nog niet aan bod is gekomen ten tijde dat de opgave
gemaakt wordt mag t\'{o}ch gebruikt worden voor die opgaven. Dit
vergemakkelijkt de boel.

\item We worden attent gemaakt op de peer assessments en peer reviews.
Dat wil zeggen het elkaar cijfers geven en het gezamenlijk
controleren van het werk van de groepsleden.
\end{itemize}


\section{Afspraken}

\begin{itemize}
\item Roy houd de lijst bij van oplossingen (met opgaven). Deze worden
voorzien van de reviews.

\item Gemaakte opgaven dienen naar Roy gemailt te worden, Roy verspreid de
nieuwe versie(s) dan over email.

\item De planning wordt duidelijker gemaakt. Er wordt een
standaardnumering van opgaven ingevoerd. We hoeven niet meer
afhankelijk te zijn van onze eigen interpretatie van het document.

\item Er wordt een post gemaakt op Studyweb met daarin een link naar
lms.ogotue.nl en onze groep daarop. Dit is het digitaal logboek.

\item Er worden gebruiksafspraken gemaakt voor het logboek.

\item De voorzitter zal een agenda plaatsen op Studyweb.

\item Iedereen dient deze notulen door te nemen en opmerkingen te
maken voor komende discussiepunten.

\item Bij de volgende vergadering is Roy voorzitter en Giso de notulist.

\item De volgende vergadering is op 20-9-2006, om 13:30 in HG 8.58.
\end{itemize}

\section{Verdere werkverdeling}

\subsection{Voor 15-9-2006}
(deze opdrachten staan op pagina 10 en 12)

\begin{itemize}
\item 4.1.a Leroy Bakker
\item 4.1.b Roy Berkeveld
\item 4.1.c Giso Dal
\item 4.2.a Etienne van Delden
\item 4.2.b Edin Dudojevic
\item 4.2.c Nick van der Vreeken
\end{itemize}

\begin{itemize}
\item 5.1.a Nick van der Vreeken
\item 5.1.b Edin Dudojevicm
\item 5.1.c Etienne van Delden
\item 5.2.a Giso Dal
\item 5.2.b Roy Berkeveld
\item 5.2.c Leroy Bakker
\end{itemize}


\subsection{Voor 22-9-2006}
(deze opdrachten staan op pagina 13 en 15)

\begin{itemize}
\item 6.1.a Leroy Bakker
\item 6.1.b Roy Berkeveld
\item 6.1.c Giso Dal
\item 6.2.a Etienne van Delden
\item 6.2.b Edin Dudojevic
\item 6.2.c Nick van der Vreeken
\end{itemize}

\begin{itemize}
\item 7.1.a Nick van der Vreeken
\item 7.1.b Edin Dudojevic
\item 7.1.c Etienne van Delden
\item 7.2.a Giso Dal
\item 7.2.b Roy Berkeveld
\item 7.2.c Leroy Bakker
\end{itemize}

\end{document}
