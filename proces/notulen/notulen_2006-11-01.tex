\documentclass[11pt]{article}
\usepackage{geometry}
\geometry{a4paper}
\usepackage[dutch]{babel}
\usepackage{a4wide}

\title{Notulen vergadering OGO 1.1 groep 1a d.d. 2006-11-01}
\author{L.F.B.C. Bakker}
\date{}

\begin{document}

\maketitle

\section{Opening}

Edin is vandaag voorzitter en heeft netjes de agenda gemaakt. Leroy is notulist. \\
\\
Iedereen van de groep is aanwezig uitgezonderd van Nick.


\section{Nieuwe afspraken}

De groep bestaat vanaf nu uit 5 personen. Nick heeft de hoop op een
goed einde opgegeven en probeert het volgend jaar opnieuw. Dit heeft
een aantal gevolgen voor de groep. De taakverdeling moet opnieuw
worden verdeeld en iedereen moet ook meer taken gaan verrichten. \\
\\
Er moet een harde planning komen om alles voor de deadline af te
maken. Deze is al deels gemaakt door Roy, die een schema heeft
gemaakt met daarin een taakverdeling. In dit schema kunnen ook de
peer reviews worden bijgehouden.


\section{Regeling specificatieopdracht}
Er is een mogelijkheid om in januari of februari de
specificatieopdracht te maken. Dit moet nog besproken worden met
Jaap van der Woude.

\section{Regeling productdocument 1}
Productdocument 1 wordt vandaag afgerond en een voorlopige versie
wordt bij Martijn in het postvak gestopt. Het procesdocument wordt
hierbij toegevoegd.

\section{Regeling productdocument 2}
Zoals eerder vermeld is er een schema waarin taakverdeling staat en
waarin de peer reviews bijgehouden kunnen worden. Verder moeten we
elkaar meer peilen over het verloop van de taken. \\
Tijdens OGO uren moeten de peer reviews gedaan worden, omdat de
opgaven ook op andere tijdstippen gedaan kunnen worden.

\section{Mededelingen}
Peer assessments zijn aanwezig en kunnen worden ingevuld. \\
Er moeten duidelijke regelingen komen voor het bijhouden van het
logboek en voor de specificatielijst.

\section{Sluiting}
De vergadering is hierbij gesloten. De volgende vergadering is
woensdag 8 november 2006 om 15.00 uur.
\end{document}
